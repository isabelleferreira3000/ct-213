\documentclass[conference]{IEEEtran}
\IEEEoverridecommandlockouts
% The preceding line is only needed to identify funding in the first footnote. If that is unneeded, please comment it out.
\usepackage{cite}
\usepackage{amsmath,amssymb,amsfonts}
\usepackage{algorithmic}
\usepackage{graphicx}
\usepackage{textcomp}
\usepackage{xcolor}
\usepackage[brazilian]{babel}
\usepackage[utf8]{inputenc}
\usepackage[T1]{fontenc}
\usepackage{listings}
\usepackage{color}
\usepackage{float}
\usepackage{multirow}

\definecolor{dkgreen}{rgb}{0,0.6,0}
\definecolor{gray}{rgb}{0.5,0.5,0.5}
\definecolor{mauve}{rgb}{0.58,0,0.82}

\lstset{frame=tb,
  language=Java,
  aboveskip=3mm,
  belowskip=3mm,
  showstringspaces=false,
  columns=flexible,
  basicstyle={\small\ttfamily},
  numbers=none,
  numberstyle=\tiny\color{gray},
  keywordstyle=\color{blue},
  commentstyle=\color{dkgreen},
  stringstyle=\color{mauve},
  breaklines=true,
  breakatwhitespace=true,
  tabsize=3
}
\lstset{language=Python}
\def\BibTeX{{\rm B\kern-.05em{\sc i\kern-.025em b}\kern-.08em
    T\kern-.1667em\lower.7ex\hbox{E}\kern-.125emX}}
\begin{document}

\title{Relatório do Laboratório 7: \\ Redes Neurais\\
}

\author{\IEEEauthorblockN{Isabelle Ferreira de Oliveira}
\IEEEauthorblockA{\textit{CT-213 - Engenharia da Computação 2020} \\
\textit{Instituto Tecnológico de Aeronáutica (ITA)}\\
São José dos Campos, Brasil \\
isabelle.ferreira3000@gmail.com}
}

\maketitle

\begin{abstract}
Esse relatório documenta a implementação de uma rede neural de duas camadas para realizar a segmentação de cores para o futebol de robôs. Para isso, foi necessário configurar essa rede neural para realizar classificação multi-classe, implementando os algoritmos de Forward Propagation (inferência) e Back Propagation (treinamento) para essa rede.
\end{abstract}

\begin{IEEEkeywords}
Redes neurais, segmentação de cores, \textit{Forward Propagation}, \textit{Back Propagation}
\end{IEEEkeywords}

\section{Introdução}
Redes neurais são sistemas de computação que podem reconhecer padrões escondidos, agrupar dados e classificá-los, além de, com o tempo, aprender e melhorar continuamente.

\begin{figure}[htbp]
\centering
\centerline{\includegraphics[scale=0.4]{imagens/rede_neural.png}}
\caption{Exemplo de rede neural, com duas camadas (1 camada de entrada, 1 camada escondida e 1 camada de saída), como a trabalhada nesse laboratório. Essa imagem de exemplo foi apresentada no site \cite{b1}}.
\label{translated_sphere/ses}
\end{figure}

Dentre os mais diversos tipos de algoritmos de otimização, existem os métodos baseado em estratégias evolutivas, métodos esses inspirados na evolução natural das espécies. Esses métodos seguem a ideia de: dada uma população de possíveis soluções, aplica-se uma função para medir a qualidade dessas soluções candidatas. Essa qualidade quantitativa é chamade de \textit{fitness}. Com base no \textit{fitness}, é possível escolher as melhores soluções e, a partir delas, gerar mutações para se criar uma nova população. Esse processo é repetido até que a convergência chegue a um resultado satisfatório de otimização.

O pseudo-código geral de algortimos utilizando estratégias evolutivas pode ser visto a seguir. Em seguida, será apresentado como esse algoritmo foi implementado no contexto do laboratório.

\begin{lstlisting}
def evolution_strategy(J, m0, C0, hyperparams):
	mu, lambda = unwrap_hyperparams(hyperparams)
	m, C = m0, C0
	while not check_stopping_condition():
		population = multivariate_normal(m, C, lambda)
		population = sort_ascending(population, J)
		parents = population[0:mu]
		m = mean(parents)
	return population[0]
\end{lstlisting}

No pseudocódigo acima, \textit{J} é a função para medir a qualidade das soluções candidatas; \textit{m0} e \textit{C0} são a média e a matriz de covariância iniciais da população que será gerada, respectivamente; \textit{m} e \textit{C} são, de forma análoga, a média e a matriz de covariância atuais da população que será gerada, respectivamente; e \textit{mu} é o tamanho da população considerada como "melhores soluções até então".

\section{Implementação do algoritmo}
Na parte relativa a implementação do algoritmo utilizando uma simples estratégia evolutiva (SES, do inglês \textit{Simple Evolution Strategy}), era necessário preencher a função \textit{tell()} da classe \textit{SimpleEvolutionStrategy}. Recebendo os valores de \textit{fitness} encontrados na população anterior, essa função era a responsável por atualizar a média e a matriz de covariância utilizando os melhores avaliados na antiga população e também evoluir a própria população a cada iteração. Essa função a se completar estava no código base fornecido \cite{b1}.  

Além disso, era necessário comparar os desempenhos desse algoritmo SES com o algoritmo CMA-ES, já fornecido no código base, aplicando-os na otimização de quatro funções, a se saber: esfera transladada, e as fuções de Ackley, Schaffer 2 e Rastrigin 2D.

A análise de vários pontos do algoritmo descrito acima terá uma breve descrição em alto nível da sua implementação a seguir. 

Primeiramente, foi necessário ordenar a população atual tendo em vista os valores de \textit{fitness} associados a ela recebido por parâmetro. Isso foi feito utilizando a função \textit{argsort()} da biblioteca \textit{Numpy}, conforme sugerido pelo roteiro do labotarório.

Após isso, as \textit{mu} melhores amostras dessa população foram separadas para ajudar no cálculo de sua matriz de covariância e, posteriormente, realizar o cálculo da nova média das melhores amostras dessa geração em questão. Essa matriz de covariância foi feita a partir da multiplicação de uma determinada matriz auxiliar pela sua transposta, sendo essa matriz auxiliar a diferença entre essas melhores amostras e a média encontrada na população anterior.

Dessa forma, tendo em posse a nova matriz de covariância e a nova média das melhores amostras da população anterior, é possível gerar uma nova população, evoluindo para a próxima geração de possíveis candidatas a solução.

Para testar o funcionamento dessa implementação, foram alterados os valores das variáveis \textit{algorithm} (entre "ses" e "cmaes") e \textit{function} (entre as funções \textit{translated\underline{\space}sphere}, \textit{ackley}, \textit{schaffer2d}, \textit{rastrigin}) no arquivo \textit{test\underline{\space}evolution\underline{\space}strategy.py} do código base, gerando imagens dos resultados da otimização de cada uma dessas funções usando a estratégia evolutiva escolhida.

Por fim, a fim de realizar o \textit{benchmark} através de simulações de Monte Carlo para comparar os desempenhos dessa estratégia evolutiva simples e do CMA-ES, foram alterados novamente os valores das variáveis \textit{algorithm} e \textit{function}, dessa vez no arquivo \textit{benchmark\underline{\space}evolution\underline{\space}strategy.py} do código base, gerando dessa vez os gráficos comparativos de rendimento para diferentes situações de SES e CMA-ES. Esses gráficos foram apresentados nas Figuras \ref{translated_sphere/mean_fitness} a \ref{rastrigin/best_fitness}.

\section{Resultados e Conclusões}

\subsection{Teste das Estratégias Evolutivas}

A otimização para testar a implementação foi executada para as quatro funções evolutivas já citadas na seção anterior, cuja equação matemática se encontra no roteiro do laboratório \cite{b1}. Os resultados dessas execuções foram satisfatórios e saíram conforme o esperado, comprovando o correto funcionamento da implementação e a validade da utilização das estratégias evolutivos na otimização de funções. Esses resultados foram apresentados nas Figuras de \ref{translated_sphere/ses} a \ref{rastrigin/cmaes}.

Vale reparar que os resultados de ambas implementações foram bastante equivalentes, com exceção do caso da função Rastrigin, no qual cada algoritmo convergiu em míminos locais diferentes.

\begin{figure}[htbp]
\centering
\centerline{\includegraphics[scale=0.5]{imagens/translated_sphere/ses.png}}
\caption{Otimização da função de Esfera Transladada usando a estratégia evolutiva SES. O resultado encontrado é o ponto vermelho.}
\label{translated_sphere/ses}
\end{figure}

\begin{figure}[htbp]
\centering
\centerline{\includegraphics[scale=0.5]{imagens/translated_sphere/cmaes.png}}
\caption{Otimização da função de Esfera Transladada usando a estratégia evolutiva CMA-ES. O resultado encontrado é o ponto vermelho.}
\label{translated_sphere/cmaes}
\end{figure}

\begin{figure}[htbp]
\centering
\centerline{\includegraphics[scale=0.5]{imagens/ackley/ses.png}}
\caption{Otimização da função de Ackley usando a estratégia evolutiva SES. O resultado encontrado é o ponto vermelho.}
\label{ackley/ses}
\end{figure} 

\begin{figure}[htbp]
\centering
\centerline{\includegraphics[scale=0.5]{imagens/ackley/cmaes.png}}
\caption{Otimização da função de Ackley usando a estratégia evolutiva CMA-ES. O resultado encontrado é o ponto vermelho.}
\label{ackley/cmaes}
\end{figure}

\begin{figure}[htbp]
\centering
\centerline{\includegraphics[scale=0.5]{imagens/schaffer2d/ses.png}}
\caption{Otimização da função de Schaffer Nº 2 usando a estratégia evolutiva SES. O resultado encontrado é o ponto vermelho.}
\label{schaffer2d/ses}
\end{figure} 

\begin{figure}[htbp]
\centering
\centerline{\includegraphics[scale=0.5]{imagens/schaffer2d/cmaes.png}}
\caption{Otimização da função de Schaffer Nº 2 usando a estratégia evolutiva CMA-ES. O resultado encontrado é o ponto vermelho.}
\label{schaffer2d/cmaes}
\end{figure}

\begin{figure}[htbp]
\centering
\centerline{\includegraphics[scale=0.5]{imagens/rastrigin/ses.png}}
\caption{Otimização da função Rastrigin (2D) usando a estratégia evolutiva SES. O resultado encontrado é o ponto vermelho.}
\label{rastrigin/ses}
\end{figure} 

\begin{figure}[htbp]
\centering
\centerline{\includegraphics[scale=0.5]{imagens/rastrigin/cmaes.png}}
\caption{Otimização da função Rastrigin (2D) usando a estratégia evolutiva CMA-ES. O resultado encontrado é o ponto vermelho.}
\label{rastrigin/cmaes}
\end{figure}

\subsection{Benchmark das Estratégias Evolutivas}

A otimização para realizar o \textit{benchmark} do desempenho de cada um dos métodos evolutivos estudados (SES e CMA-ES) para diferentes funções e parâmetros foi executada para as quatro funções evolutivas já citadas. Os resultados dessas execuções demonstraram comportamentos coerentes e foram apresentados nas Figuras de \ref{translated_sphere/ses} a \ref{rastrigin/cmaes}.

É possível salientar que, embora sempre tenha havido convergências após um número significativo de iterações, nem todas as convergências chegaram a mínimos locais, como por exemplo para a função da Esfera Transaladada, que só possui um mínimo local (que também é global), mas cada situação testada chegou a valores diferentes de \textit{fitness}.

Além disso, notou-se que, para estratégias evolutivas mais simples, é necessário um número cada vez maior de elementos na população para que os resultados se tornem cada vez mais otimizados. Já para o algoritmo CMA-ES precisou de cerca de 1/4 de população para alcançar resultados similares aos do SES. Isso aconteceu para todos os casos com exceção à função Schaffer Nº 2, que estratégias mais simples e com menores populações se saíram melhor em desempenho do que CMA-ES.

\begin{figure}[htbp]
\centering
\centerline{\includegraphics[scale=0.5]{imagens/translated_sphere/mean_fitness.png}}
\caption{Evolução (e convergência) dos valores médios de fitness das populações em cada iteração para diferentes métodos evolutivos e parâmetros para a função Esfera Transladada.}
\label{translated_sphere/mean_fitness}
\end{figure}

\begin{figure}[htbp]
\centering
\centerline{\includegraphics[scale=0.5]{imagens/translated_sphere/best_fitness.png}}
\caption{Evolução (e convergência) dos melhores valores de fitness das populações em cada iteração para diferentes métodos evolutivos e parâmetros para a função Esfera Transladada.}
\label{translated_sphere/best_fitness}
\end{figure}

\begin{figure}[htbp]
\centering
\centerline{\includegraphics[scale=0.5]{imagens/ackley/mean_fitness.png}}
\caption{Evolução (e convergência) dos valores médios de fitness das populações em cada iteração para diferentes métodos evolutivos e parâmetros para a função Ackley.}
\label{ackley/mean_fitness}
\end{figure}

\begin{figure}[htbp]
\centering
\centerline{\includegraphics[scale=0.5]{imagens/ackley/best_fitness.png}}
\caption{Evolução (e convergência) dos melhores valores de fitness das populações em cada iteração para diferentes métodos evolutivos e parâmetros para a função Ackley.}
\label{ackley/best_fitness}
\end{figure}

\begin{figure}[htbp]
\centering
\centerline{\includegraphics[scale=0.5]{imagens/schaffer2d/mean_fitness.png}}
\caption{Evolução (e convergência) dos valores médios de fitness das populações em cada iteração para diferentes métodos evolutivos e parâmetros para a função Schaffer Nº 2.}
\label{schaffer2d/mean_fitness}
\end{figure}

\begin{figure}[htbp]
\centering
\centerline{\includegraphics[scale=0.5]{imagens/schaffer2d/best_fitness.png}}
\caption{Evolução (e convergência) dos melhores valores de fitness das populações em cada iteração para diferentes métodos evolutivos e parâmetros para a função Schaffer Nº 2.}
\label{schaffer2d/best_fitness}
\end{figure}

\begin{figure}[htbp]
\centering
\centerline{\includegraphics[scale=0.5]{imagens/rastrigin/mean_fitness.png}}
\caption{Evolução (e convergência) dos valores médios de fitness das populações em cada iteração para diferentes métodos evolutivos e parâmetros para a função Rastrigin (2D).}
\label{rastrigin/mean_fitness}
\end{figure}

\begin{figure}[htbp]
\centering
\centerline{\includegraphics[scale=0.5]{imagens/rastrigin/best_fitness.png}}
\caption{Evolução (e convergência) dos melhores valores de fitness das populações em cada iteração para diferentes métodos evolutivos e parâmetros para a função Rastrigin (2D).}
\label{rastrigin/best_fitness}
\end{figure}

Tendo em vista o que foi apresentado, pode-se notar, por fim, que esses algoritmos realmente se demonstraram eficazes em encontrar parâmetros otimizados para uma determinada função.

\begin{thebibliography}{00}
\bibitem{b1} M. Maximo, ``Roteiro: Laboratório 5 - Estratégias Evolutivas''. Instituto Tecnológico de Aeronáutica, Departamento de Computação. CT-213, 2019.
\end{thebibliography}

\end{document}
