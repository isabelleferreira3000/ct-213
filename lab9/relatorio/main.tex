\documentclass[conference]{IEEEtran}
\IEEEoverridecommandlockouts
% The preceding line is only needed to identify funding in the first footnote. If that is unneeded, please comment it out.
\usepackage{cite}
\usepackage{amsmath,amssymb,amsfonts}
\usepackage{algorithmic}
\usepackage{graphicx}
\usepackage{textcomp}
\usepackage{xcolor}
\usepackage[brazilian]{babel}
\usepackage[utf8]{inputenc}
\usepackage[T1]{fontenc}
\usepackage{listings}
\usepackage{color}
\usepackage{float}
\usepackage{multirow}
\usepackage{hyperref}

\definecolor{dkgreen}{rgb}{0,0.6,0}
\definecolor{gray}{rgb}{0.5,0.5,0.5}
\definecolor{mauve}{rgb}{0.58,0,0.82}

\lstset{frame=tb,
  language=Java,
  aboveskip=3mm,
  belowskip=3mm,
  showstringspaces=false,
  columns=flexible,
  basicstyle={\small\ttfamily},
  numbers=none,
  numberstyle=\tiny\color{gray},
  keywordstyle=\color{blue},
  commentstyle=\color{dkgreen},
  stringstyle=\color{mauve},
  breaklines=true,
  breakatwhitespace=true,
  tabsize=3
}
\lstset{language=Python}
\def\BibTeX{{\rm B\kern-.05em{\sc i\kern-.025em b}\kern-.08em
    T\kern-.1667em\lower.7ex\hbox{E}\kern-.125emX}}
\begin{document}

\title{Relatório do Laboratório 9: \\ Redes Neurais Convolucionais\\
}

\author{\IEEEauthorblockN{Isabelle Ferreira de Oliveira}
\IEEEauthorblockA{\textit{CT-213 - Engenharia da Computação 2020} \\
\textit{Instituto Tecnológico de Aeronáutica (ITA)}\\
São José dos Campos, Brasil \\
isabelle.ferreira3000@gmail.com}
}

\maketitle

\begin{abstract}
Esse relatório documenta a implementação, treino e teste da rede neural LeNet-5 usando o \textit{dataset} MNIST, que consiste num conjunto grande de imangens anotadas de dígitos decimais escritos à mão. Assim, será reproduzido um trabalho clássico da literatura de Redes Neurais Convolucionais (CNNs), que foi realizado originalmente por Yann LeCun.
\end{abstract}

\begin{IEEEkeywords}
\textit{LeNet-5}, \textit{MNIST}, Redes Neurais Convolucionais, Keras, Tensorflow
\end{IEEEkeywords}

\section{Implementação}
Para a implementação da rede neural conforme os parâmetros requisitados pelo roteiro do laboratório \cite{roteiro}, era necessário utilizar do código de adição de camadas a uma rede, apresentado na seção Dicas do roteiro do laboratório \cite{roteiro}. 

Tendo em vista que em \textit{keras.activations} não há função de ativação Leaky ReLU, utilizou-se a recomendação sugerida pelo roteiro \cite{roteiro} para usar Leaky ReLU no Keras, ou seja, foi adicionado uma camada do tipo LeakyRelu após ter definido uma camada (usando função de ativação linear).

Além disso, para configurar o modelo, setou-se o parâmetro \textit{loss} da função \textit{compile()} para \textit{losses.mean\underline{\space}squared\underline{\space}error}, uma vez que foi utilizado erro quadrático.

Por fim, para treinar, o tamanho do \textit{batch} foi o tamanho total de entradas, para que seja usado todo o \textit{dataset} em cada iteração do treinamento.

\begin{figure}[htbp]
%\centering
\centerline{\includegraphics[scale=0.25]{imagens/acc.png}}
\caption{Arquitetura da rede neural usada para o imitation learning. Essa imagem foi apresentada no roteiro \cite{roteiro}}.
\label{acc}
\end{figure}

\section{Resultados e Conclusões}

\subsection{Estudo de implementação de Rede Neural com Keras}
O código do arquivo \textit{test\underline{\space}keras.py} foi estudado para se entender a utilização do \textit{framework} Keras na implementação de redes neurais. O que foi aprendido resultou no texto escrito na Introdução.

\subsection{Análise do efeito de Regularização}
Esse arquivo \textit{test\underline{\space}keras.py} continha a implentação do aprendizado das funções "soma maior que zero" e "xor" para diferentes valores de regularização.

Após a execução desse arquivo, obteve-se os resultados apresentados nas Figuras de \ref{sum_gt_zero/lambda_zero/dataset_sgz} a \ref{xor/nn_classification_xor}, sendo as Figuras \ref{sum_gt_zero/lambda_zero/dataset_sgz} e \ref{xor/lambda_zero/dataset_xor} os \textit{dataset} utilizados para as funções "soma maior que zero" e "xor", respectivamente.

Analisando os resultados, é possível notar que em todas as situações (com e sem regularização, e para as duas funções) a rede obtive uma classificação satisfatória. A classificação com regularização, entretando, apresentou-se muito mais acertiva, conforme se pode observar pelas comparação das Figuras \ref{sum_gt_zero/lambda_zero/nn_classification_sgz} e \ref{sum_gt_zero/nn_classification_sgz} para "soma maior que zero" e \ref{xor/lambda_zero/nn_classification_xor} e \ref{xor/nn_classification_xor} para "xor". 

A questão da convergência da função de custo também pode ser comparada. Para os casos com regularização, a convergência se deu bem antes em número de épocas. Isso pode ser observado nas Figuras \ref{sum_gt_zero/lambda_zero/convergence_sgz} e \ref{sum_gt_zero/convergence_sgz}, para "soma maior que zero" e nas Figuras \ref{xor/lambda_zero/convergence_xor} e \ref{xor/convergence_xor} para "xor". 

\begin{figure}[htbp]
\centering
\centerline{\includegraphics[scale=0.25]{imagens/loss.png}}
\caption{\textit{Dataset} utilizado para o aprendizado da função $soma > 0$}.
\label{loss}
\end{figure}

\begin{figure}[htbp]
\centering
\centerline{\includegraphics[scale=0.25]{imagens/val_acc.png}}
\caption{Convergência da função de custo para a função $soma > 0$, sem regularização.}.
\label{val_acc}
\end{figure}

\begin{figure}[htbp]
\centering
\centerline{\includegraphics[scale=0.25]{imagens/val_loss.png}}
\caption{Resultado da classificação por rede neural para a função $soma > 0$, sem regularização.}
\label{val_loss}
\end{figure}

\begin{figure}[htbp]
\centering
\centerline{\includegraphics[scale=0.5]{imagens/test_image_3758.png}}
\caption{Convergência da função de custo para a função $soma > 0$, com regularização.}
\label{test_image_3758}
\end{figure}

\begin{figure}[htbp]
\centering
\centerline{\includegraphics[scale=0.5]{imagens/test_image_4449.png}}
\caption{Resultado da classificação por rede neural para a função $soma > 0$, com regularização.}
\label{test_image_4449}
\end{figure}

\begin{figure}[htbp]
\centering
\centerline{\includegraphics[scale=0.5]{imagens/test_image_1798.png}}
\caption{\textit{Dataset} utilizado para o aprendizado da função $xor$.}
\label{test_image_1798}
\end{figure} 

\begin{figure}[htbp]
\centering
\centerline{\includegraphics[scale=0.5]{imagens/test_image_2082.png}}
\caption{Convergência da função de custo para a função $xor$, sem regularização.}
\label{test_image_2082}
\end{figure}

\begin{figure}[htbp]
\centering
\centerline{\includegraphics[scale=0.5]{imagens/test_image_2439.png}}
\caption{Resultado da classificação por rede neural para a função $xor$, sem regularização.}
\label{test_image_2439}
\end{figure}

\begin{figure}[htbp]
\centering
\centerline{\includegraphics[scale=0.5]{imagens/misclassified_image_8.png}}
\caption{Convergência da função de custo para a função $xor$, com regularização.}
\label{misclassified_image_8}
\end{figure}

\begin{figure}[htbp]
\centering
\centerline{\includegraphics[scale=0.5]{imagens/misclassified_image_92.png}}
\caption{Resultado da classificação por rede neural para a função $xor$, com regularização.}
\label{misclassified_image_92}
\end{figure}

\subsection{Imitation Learning}
Após a implementação da rede neural com Keras conforme o explicado na seção Implementação, os resultados obtidos estão apresentados nas Figuras de \ref{imitation_learning/rightAnklePitch} a \ref{imitation_learning/rightKneePitch}. A comparação entre os gráficos de azul (curva original) e laranja (função aprendida por rede neural) dessas figuras demonstra que a implementação aconteceu de maneira satisfatória, uma vez que as funções ficaram bastante semelhantes.

\begin{figure}[htbp]
\centering
\centerline{\includegraphics[scale=0.5]{imagens/misclassified_image_18.png}}
\caption{Resultado da classificação por rede neural para a função \textit{soma > 0}.}
\label{misclassified_image_18}
\end{figure} 

\begin{figure}[htbp]
\centering
\centerline{\includegraphics[scale=0.5]{imagens/misclassified_image_247.png}}
\caption{Resultado do \textit{Imitation learning} para Right Ankle Roll.}
\label{misclassified_image_247}
\end{figure}

\begin{figure}[htbp]
\centering
\centerline{\includegraphics[scale=0.5]{imagens/misclassified_image_321.png}}
\caption{Resultado do \textit{Imitation learning} para Right Hip Pitch.}
\label{misclassified_image_321}
\end{figure}

Tendo em vista o que foi apresentado, pode-se notar, por fim, que algoritmos de \textit{Deep leaning} e o \textit{framework} Keras realmente se demonstraram eficazes em realizar aprendizado por imitação.

\begin{thebibliography}{00}
\bibitem{roteiro} M. Maximo, ``Roteiro: Laboratório 8 - Imitation Learning com Keras''. Instituto Tecnológico de Aeronáutica, Departamento de Computação. CT-213, 2019.
\end{thebibliography}

\end{document}
